\documentclass[12pt]{article}
\usepackage{palatino}
\begin{document}
\fontsize{12}{16}
\selectfont

\centerline{\large INSTRUCTIONS ON USING}
\centerline{\large The ValidFX Application}

\vspace{0.2 in}

\begin{enumerate}
\item  Click the file explorer icon at the launch bar. The launch bar is located at the bottom
 of the screen. 

\item  Change to the Documents folder by double clicking on the {\sl Documents} folder
  or whichever other folder where the EXCEL documents are located.

\item  Locate the EXCEL file you wish to work on. Double click on the file icon. The file should open in
EXCEL

\item  Work on the file as usual.

\item  Save your work by clicking on \fbox{File}  $\longrightarrow$ \fbox{Save} in the menu.

\item  To print the VALID TICKETS go to the menu and click 
on \fbox{File}  $\longrightarrow$ \fbox{Save as}.

\item  Locate the Documents Folder icon. Click on it. The {\sl Save As} dialog should appear.

\item  In the dialog, locate the {\sl Save as type} selection box. 
Click on the down-arrow symbol, $\vee$,
  located in the extreme right of the selection box.

\item  Choose {\sl Text (Tab delimited) *.txt}.

\item  On the File name selection box also click the down-arrow $\vee$ and choose\hfill\break 
       "C:$\backslash$User$\backslash$USer$\backslash$Documents$\backslash$data.txt"

\item  Click on the \fbox{Save} button. A {\sl Confirm Save As} dialog box should appear. 
The dialog box will 
  ask that you confirm that you want to replace the {\bf data.txt}
  file that already exists. Click on \fbox{Yes}.

\item  Microsoft Excel will show a dialog box warning that the 
selected file type does not support workbooks
that contain multiple sheets. Click \fbox{OK} to confirm you understand but you wish to continue.

\item Another Microsoft Excel dialog box will appear warning 
that some features in your workbook might be lost
under the {\bf Text (Tab delimited)} format. Click \fbox{Yes} to confirm that 
you wish to keep using that format.

\item  Close the open file by clicking \fbox{File}  $\longrightarrow$ \fbox{Close}.

\item  Microsoft Excel will display a dialog box asking if you wish to save your changes to 
{\bf data.txt}.
Click on the button \fbox{Don't Save}.

\item You may now close Excel by clicking 
the {\sl Close window} button located on the upper right corner of the
Microsoft Excel window. The {\sl Close window} button looks like a large \fbox{X}.

\item On the desktop, locate the icon named {\bf ValidFX}. Double click on it.

\item ValidFX will open a command window---which you should ignore---and 
the {\bf Create Valids JavaFX Version}
application.

\item Click on the \fbox{Browse} button. Locate the Documents folder and double click on it.

\item Locate the file {\bf data.txt}, select it by clicking on it once.

\item Confirm that {\bf data.txt} is displayed in the {\sl File name} textbox. 
Click the \fbox{Open} button.

\item Proceed as usual by entering the proper values in the {\sl Target Field} 
and the {\sl Target Value} in the ValidFX
application. Click the \fbox{OK} button when satisfied with the values entered.

\item {\sl Notepad} should open with the formatted Valids tickets.

\item Click on \fbox{File}  $\longrightarrow$ \fbox{Page Setup}. The Page Setup dialog window should open.

\item Locate the {\sl Size} selection box. Choose {\bf Legal (215.9 x 355.6)}. 
Leave the {\sl Source} selection box
to Auto. Orientation should be {\sl Portrait}.

\item Change the margins to the following values:

\begin{tabular}[t]{ll}
Left   &  20 \\
Right  &  5  \\
Top    &  8 \\
Bottom &  20 \\
\end{tabular}

\item Click \fbox{OK}.

\item To print the Valids Ticket, go to the menu bar and select 
\fbox{File}  $\longrightarrow$ \fbox{Print}.

\item To close the ValidFX Application click on the button \fbox{Close}.

\end{enumerate}

\end{document}






. 







